
\documentclass[10pt,twocolumn]{article}
\usepackage[margin=1.75cm]{geometry} 
\usepackage[parfill]{parskip}
\usepackage{graphicx}

\title{Notes on Fitting NAF}
\author{David Champredon}


%%% BEGIN DOCUMENT
\begin{document}

\maketitle


\section{Age distributions conditional on household size}

We could not have a direct access to the distribution of age of individuals conditional on their household size. But we do have access to the age distribution of the entire population and also the size distribution of households. 
Hence, it is possible to reconstruct (unfortunately, not in a unique fashion) from those data the distribution of age of individuals conditional on their household size.


Let $\phi_{i,j}$ be the distribution of age for the $i$th oldest individual in a household of size $j$. All distributions $\phi$ are parametrized as Beta distributions. The parameters are fitted such that, combined with the distribution of household size, they give a age distribution for the whole population (\emph{not} conditional on household size) the closest to the data.

The fit is performed using an optimization procedure (\texttt{nlopt} or ABC in R).

\section{Frailty index}

The shape of the frailty index as a function of age is calibrated on data of the proportion of vulnerable individuals (pre-existing condition as defined by PHAC, ie cardiovascular, respiratory conditions).

The fit is performed using an optimization procedure (\texttt{optim} in R, though I manage to have a better fit manually!).


\section{Probability of hospitalization following influenza infection}

Given a symptomatic influenza infection, the probability to be hospitalized is 
$$p_H = \alpha f $$
where $f$ is the individual's frailty and $\alpha$ the parameter to fit to data. 

The data used is the proportion of hospitalization caused by influenza per 100,000 population, by age group. The shape of the hospitalization probability as a function of age is driven by the frailty index. Hence $\alpha$ only affects the overall level of $p_H$.

The fit is done manually: choose a value for $\alpha$ such that the probabilities for all age group fall in the range of value collected over several years (which are pretty stable\cite{Schanzer:2013dn}). See Figure \ref{fig:probaHospDeath}.


So it's not really a ``fit'', more a value that is set arbitrarily, but based on existing data.

\begin{figure}[htbp]
   \centering
   \includegraphics[width=0.49\textwidth]{proba-hosp-death-ontario.pdf} % requires the graphicx package
   \caption{Rate of hospitalization and death caused by influenza for Ontario.}
   \label{fig:probaHospDeath}
\end{figure}


\section{Probability of death after hospitalization}


TO DO


\section{Basic reproductive number}


\textbf{Two methods: must get my head around those and choose! both methods can give quite different values...}

\subsection*{Counting initial secondary cases}
The basic reproductive number $R_0$ is calculated by averaging the number of secondary cases of individuals infected in the first 2 days after the epidemic starts. 
The choice of 2 days is driven by the fact the generation interval is smaller than 2 days with the natural history parameters taken for influenza in this study. Hence, it is likely (but not guaranteed) that the secondary cases recorded in this short period of time are made among a fully susceptible population. 
Moreover, considering this time window gives less noisy results for $R_0$ than calculating it from the index case(s) only.

\subsection*{Calculating $R_0$ implied from an SIR}
I believe $R_0$ is model dependent. Most of $R_0$ estimations have been done with some kind of SIR model, so the $R_0$ values to fit must be compared to a SIR.
The methodology is as follows: calculate the mean incidence time series $I_{k,t}$ from each of the $k=1,...,n$ Monte Carlo iterations of the agent-based model. 
We can approximate $R_0$ with a naive model
$$R_0 \sim 1 + r G$$
where $r$ is the incidence exponential growth rate and $G$ the mean generation interval (assumed known). For $t$ near 0, we have  $I_{k,t}\sim i_0 e^{r_kt}$, so the estimate $\hat{r}_k$ is obtained with a linear regression on the log incidence:
$$ \log(I_{k,t}) = \log(i_0) + \hat{r}_k t$$
The estimate of $R_{0k}$ (in a SIR world) for the $k$th MC iteration is given by substituting:
$$\hat{R_0}_k = 1 + \hat{r}_k G$$
Finally, the overall estimate is simply the average:
$$\hat{R}_0 = \frac{1}{n}\sum_{k=1}^n \hat{R_0}_k$$


Alternative (which seems better): do the same, but on the average across all MC iteration (not calculating $r_k$ on each MC iteration)



\end{document}






