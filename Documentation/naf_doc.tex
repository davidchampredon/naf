\documentclass[11pt, onecolumn]{article}
% \setlength{\columnsep}{0.2cm}

\usepackage{amssymb}
\usepackage{amsmath}
\usepackage[a4paper, margin=2.25cm]{geometry}
%\usepackage{mathtools}
\usepackage{graphicx}

\usepackage{color}
\usepackage[usenames,dvipsnames,svgnames,table]{xcolor}

\usepackage[colorlinks=true,linkcolor=blue, citecolor=BlueGreen]{hyperref}
%\hypersetup{ pdfborder = {0 0 0}}

\usepackage[parfill]{parskip}
\usepackage{lscape}

\usepackage{xspace}
\usepackage[labelfont=bf,font={small,sl},labelsep=period,justification=raggedright]{caption}

\usepackage{fancyhdr}
\pagestyle{fancy}
\lhead{ }
\rhead{\naf documentation}

%%%%%%%%%%%%%%%%%%%%%%%%

\title{\naf (New Agent-based model for Flu) Documentation}
\author{David Champredon}

%%%%%%%%%%%%%%%%%%%%%%%%

\newcommand{\ttt}[1]{\texttt{#1}}
\newcommand{\one}[1]{\textbf{\large{1}}_{#1}}
\newcommand{\warning}[1]{\textbf{\textcolor{OrangeRed}{#1}}}
\newcommand{\note}[1]{\textit{\textcolor{Grey}{Note: #1}}}
\newcommand{\eg}{\textit{e.g.}\xspace}
\newcommand{\ie}{\textit{i.e.}\xspace}
\newcommand{\naf}{\textsf{NAF}\xspace}


%%%%%%%%%%%%%%%%%%%%%%%%
%%%%%%%%%%%%%%%%%%%%%%%%
%%%%%%%%%%%%%%%%%%%%%%%%
%%%%%%%%%%%%%%%%%%%%%%%%

\begin{document}
\maketitle

\vspace{1cm}

\tableofcontents

\newpage

\section{Introduction}

This documentation describes the implementation of an agent-based model, named \naf, designed specifically for the spread of influenza at the population level. 

The implementation of \naf is \emph{not} event driven. Instead, time is divided in relatively coarse ``slices'' that represent relevant epidemiological periods of a day. Within a given time slice, epidemiological events are calculated only \emph{once}. This approximation allows significant computing time saving for an acceptable loss of realism (of course that depends on the size of time slices). Although each individual is uniquely identified, the information of who infects who is omitted for performance (this is optional though; and time of transmission is always recorded to keep track of the generation interval). 


\naf is implemented with a representation of a spatial structure. The fundamental spatial unit is called a ``social place'' which broadly represents any physical place human mix. Because it is preferable to have a limited number of social places types, only the most relevant ones on which data exists are explicitly represented (for example households, schools, workplace, etc. ). A social place type that represents all other types not explicitly modelled is used. 
In order to have a similar hierarchical structure as real world data, social places are gathered in ``area units'', which are themselves gathered in ``regions''.

The programming language for the model is C++. The executable is wrapped in R, mostly for output analysis and debugging.


\section{Population}

\subsection{Social place}

\naf is implemented with a spatial structure, and its fundamental spatial unit is called a ``social place''. It represents real world locations where individuals may have relevant contacts for disease transmission (\eg households, schools, public transportations, etc). Because all types of such social places cannot be explicitly described in the model, a social place typed ``other'' is used for all other real-world locations.
In order to have a similar hierarchical structure as real world data, social places are gathered in ``area units'', which are themselves gathered in ``regions''.

The size of a social place, that is the number of linked individuals, is pre-specified with a distribution given as an input. The size distribution is typically a histogram taken from national statistics, when available. 




\subsection{Individuals}

Individuals are uniquely identified in the model with an identification number (ID). The modelled features of an individual are its age, immunity to the disease and frailty. Immunity and frailty are modelled as real number between 0 and 1. They determine respectively the risk of acquiring the disease and of hospitalization. Immunity and frailty can be changed by vaccination and  treatment.

These individuals are ``linked'' to social places. A link means that the individual will eventually visit and have contacts in this social place (according to its schedule, see details later).
All individuals are linked to a social place of type ``household''. Link to other social places depends on individuals: for example, all children aged between 3 and 18 years old are linked to a school but none is linked to a workplace (Figure \ref{fig:SP_indiv}. 


\begin{figure}[!ht]
\centering
    \includegraphics[angle=0,width=0.99\textwidth]{figures/SP_indiv.png}
\caption{Individual and their linked social places. This is a simple illustration of how linkage between individuals and social places is modelled. All individuals are linked to social places of type ``household'' (solid link). Links to other social places depends on the individual's features, like age. For rxample, individual 1 is an employed adult and hence is linked to a workplace, but not a school.}
\label{fig:SP_indiv}
\end{figure}


\subsection{Population construction}

The world where the simulations will take place is defined iteratively. Briefly, a large number of individuals are created and assigned to households according to the pre-specified distribution of sizes. All household have at least one individual. Once all the households are populated, individuals are assigned an age based on pre-specified age distributions conditional on the household size. For example, a household of size one is associated with an age distribution that has a lower value of 18 years old in order to avoid households populated with only one child. There cannot be an empty household or an individual without a link to a household.

Construction of other social places is more straightforward. Individuals are assigned randomly to workplaces, schools, public transportation, etc. based on their features (\eg age, distance from household \note{not implemented yet}). 

Only social places tagged as ``other'' do not have individuals linked to them, because their purpose is to be visited randomly. 



\section{Simulator}

\subsection{Schedules and Spatial movements}

Every individual is assigned a daily ``schedule'' that will determine which social place the individual should visit at a given time of the day. There is a pre-specified probability that the visit actually occurs (based on the individual features, \eg symptomatic infection will reduce the probability \note{not fully implemented yet}).

There is little mixing if an individual would just go back and forth between its household and workplace for example. But social places like ``other'' or ``public transportation'' are supposed to bring some external mixing.


\begin{figure}[!ht]
\centering
    \includegraphics[angle=0,width=0.99\textwidth]{figures/schedule.png}
\caption{Schedules. Simple illustration of different schedules for two individuals. In this example, a day is divided into 4 unequal time slices. Each time slice has the same duration for all individuals, but the social place destination can be different. For example, in this illustration individual $j$ visits its household and school for the first two time slices of the day. Then visit an other social place, randomly chosen. Then finishes the day backin its household.}
\label{fig:SP_indiv}
\end{figure}


\subsection{Transmission process}

Individuals visit a social place during a period that is determined by the time slice of their schedule. During this visit, interaction with other individuals are modelled only if it involves a potential transmission. 

When an infectious individual is present in a social place, a pre-specified contact rate will determine the number of contacts it will have during its time in that social place. The contact rate is drawn from a Poisson distribution with a density itself also drawn from an exponential distribution. The fact the intensity is exponentially distributed allows for super-spreading events. The intensity distribution is also parameterized based on the features of individual and of the social place (for example, larger for young ages and in public transports).

Once the number of contacts from an infectious individual is determined, susceptible individuals in this social place are selected randomly as its contacts. The probability of actual transmission is the product of two probabilities. The probability the contact from the infectious individual is infectious enough depends on the symptomatic status. The probability the susceptible contact acquires the disease is based on its immunity index. The probability the infection will be symptomatic can be a function of both frailty and immunity (\note{exact functional form not decided yet}). See Figure \ref{fig:transmission}.


\begin{figure}[!ht]
\centering
    \includegraphics[angle=0,width=0.99\textwidth]{figures/transmission.jpg}
\caption{Transmission. Simplified illustration of the transmission process for an infectious individual in a given social place. DoL: duration of latency; DoI: duration of infectiousness.}
\label{fig:transmission}
\end{figure}


\subsection{Disease progression}

Once a individual has acquired the disease, durations of latency and infectiousness are drawn from a probability distribution. Hospitalization, the duration of hospitalization and disease-induced death are also determined stochastically at acquisition time.
Probabilities to be hospitalized and disease-induced death are based on frailty only.
Deciding these events and durations at acquisition time is more computationally efficient than at each time step of the simulation.

If an individual is hospitalized, it is moved to its linked social place labelled ``hospital'' and stays there for the duration of its hospitalization (movements to other social places directed by its schedule are ignored during hospitalization). Disease-induced death can only occur for hospitalized individuals.



\section{Interventions}

Antiviral treatment and vaccination are implemented as part of ``interventions''. There can be several simultaneous interventions of different nature.
Effect of treatment and vaccination is modelled based on a summary of evidence (see section \ref{sec:antiviral})

Treatment reduces the duration of infection and has no impact on frailty or immunity. Vaccine decreases frailty and increases immunity, both stochastically. The change on both immunity and frailty triggered by vaccination has a (stochastic) time lag and their values change exponentially to the new values drawn at time of vaccination.

\note{prophylactic treatment will be implemented if needed.}





\section{Literature on influenza disease parameters}

\subsection{Natural history}

\textbf{Incubation.} Typically 1-4 days.

\textbf{Viral shedding.} Typically 1 day before symptoms appear. Could be sooner for young children. Shedding sharply reduces after 3-5 days, but could take much longer in young children (up to 10 days after onset). 

\textbf{Recovery.} Typically after 3-7 days after onset.


\subsection{Viral shedding}

Shedding can start about 1 day before symptoms onset, but can even be earlier among young children \cite{CDC:2011wq}. Shedding lasts for several days but is usually sharply reduced by day 3-5 after onset \cite{CDC:2011wq}. The decrease rate is exponential and weak patients experience a slower and longer decrease \cite{Lee:2009dc}.

\begin{figure}[!ht]
\centering
    \includegraphics[angle=0,width=0.6\textwidth]{figures/VL_Lee.png}
\caption{Temporal evolution of viral load. Solid lines represent patients with major commorbidities, dashed those without. Source: Lee 2009 \cite{Lee:2009dc}}
\label{fig:VL_Lee}
\end{figure}


\subsection{Hospitalization}

Large majority of children hospitalizations last less than 2 days. Median 3-4 days. Hospitalization mostly occur in young children with pre-existing condition.

Fatality rate in hospital is typically 1-3\% in wealthy countries, but is higher among elderly (6-30\%, but this estimates include poor countries \cite{Wong:2015bb}.



\subsection{Antiviral Treatments}
\label{sec:antiviral}
Ressources:
\begin{itemize}
\item  CDC recommendations offer a nice overview \cite{CDC:2011wq}.
\item Reviews: Ebell\cite{Ebell:2014ic},  Jefferson 2014 \cite{Jefferson:2014ei}, Jackson 2011 \cite{Jackson:2011ff}
\end{itemize}

Four drugs: 
\begin{itemize}
\item Amantadine and rimantadine: for influenza A only, experience drug resistance.
\item Oseltamivir, Zanamivir: for both influenza A and B, drug resistance rare.
\end{itemize}

Overall, when used on patients already infected, treatment reduces marginally symptoms duration and does not reduce the risk of hospitalization \cite{Ebell:2014ic,Jefferson:2014ei}. 
If administrated within 2 days of onset: reduce duration of uncomplicated illness by 1 day.
If administrated within 1 day of onset: reduce duration of uncomplicated illness by 3.5 days among young children($<3$ years-old) \cite{CDC:2011wq}.

Treatment reduces amount of viral shedding (by how much?) but it's not clear if it reduces the shedding duration.

It is not clear if treatment reduces the risk of severe complications.

Post-exposure chemoprophylaxis is effective at preventing illness (may still be asymptomatic?): 70-90\% reduction in households context. Typical treatment duration: 10 days after most recent contact known \cite{CDC:2011wq}.

Pre-exposure chemoprophylaxis is effective at preventing illness: 80-90\%. Duration of treatment can be as long as 1 month. But because of resistance concerns, pre-exposure chemoprophylaxis is used only for persons at very high risk \cite{CDC:2011wq}. 

Adherence to treatment -- especially for long treatment durations -- is an issue because of side effects.



\subsection{Vaccination}

Influenza vaccine seems to be effective at reducing disease symptoms and severity, if infected later (at least 10 days later?).


\section{Performance}

\note{this section is not completed...}


\begin{table}[htdp]
\caption{Computing time on one CPU (3.2 GHz, 8GB memory) }
\begin{center}
\begin{tabular}{|ccc|}
\hline
Num. Indiv. & Num. SP & time\\
\hline
4600 & 67 & 11 sec\\
9200 & 137 & 21 sec\\

\hline
\end{tabular}
\end{center}
\label{default}
\end{table}%



\section{Tests}

\note{this section is not completed...}

\begin{itemize}
\item \textbf{SEIR ODEs.} One unique spatial location, homogeneous mixing. Compare solutions of the standard ODE model with \naf. Test passed, see figure xxx.
\item \textbf{Migrations between social places.} Only two social places, no epidemic. Check if migrations occur as specified by schedule. Test passed. 
\end{itemize}


\bibliography{papers}
\bibliographystyle{plain}

\end{document}




